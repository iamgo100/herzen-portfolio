% Шаблон из ЛР2 (ЛР6)
\documentclass[a4paper,12pt]{article}
\usepackage[T2A]{fontenc}
\usepackage[utf8]{inputenc}
\usepackage[english,russian]{babel}
\usepackage{amsmath,amsfonts,amssymb,amsthm,mathtools}
\usepackage{wasysym}
\usepackage{hyperref}

\author{Елкина Галя, 2пг}
\title{Работа с текстом в \LaTeX{}}
\date{\today}

\begin{document}

\maketitle
\newpage

\section*{Задание 1}

\subsection*{Для чего предназначена издательская система LaTeX?}

\LaTeX - это издательская система, помогающая писателям сосредоточится на содержании, а не на оформлении.\\
Она решает такие задачи, как:

\begin{itemize}
\item алгоритмы расстановки переносов, определения междусловных пробелов, балансировки текста в абзацах;
\item автоматическая генерация содержания, списка иллюстраций, таблиц и т. д.;
\item механизм работы с перекрёстными ссылками на формулы, таблицы, иллюстрации, их номер или страницу;
\item механизм цитирования библиографических источников, работы с библиографическими картотеками;
\item размещение иллюстраций (иллюстрации, таблицы и подписи к ним автоматически размещаются на странице и нумеруются);
\item оформление математических формул, возможность набирать многострочные формулы, большой выбор математических символов;
\item оформление химических формул и структурных схем молекул органической и неорганической химии;
\item оформление графов, схем, диаграмм, синтаксических графов;
\item оформление алгоритмов, исходных текстов программ (которые могут включаться в текст непосредственно из своих файлов) с синтаксической подсветкой;
\item разбивка документа на отдельные части (тематические карты).
\end{itemize}

\subsection*{В каких случаях рационально её использовать?}

Такую ИИС
\footnote{Интегрированная издательская система}
стоит использовать в тех случая, когда важно не то, как выглядит текст, а то, что написано в это тексте.
То есть тогда, когда важен смысл, а не картинка.
\LaTeX прекрасно подойдет для набирания текста документов, книг или статей в газеты.

\subsection*{Какие преимущества имеет работа в этой системе?}

\begin{enumerate}
    \item Есть удобные способы набора \huge формул. \normalsize
    \item Удобные способы организации документа.
    \item \tiny Не \normalsize нужно задумывать про форматирование документа.
\end{enumerate}

\subsection*{Какие сложности могут возникнуть при работе в этой системе?}

\begin{flushleft}
При работе в \LaTeX с различными видами текста нужно также уметь работать с библиотеками.
Их нужно уметь подключать и использовать.
\end{flushleft}

\subsection*{Какие недостатки отмечают пользователи при работе с этой системой?}

Сам
\href{https://www.latex-project.org/}{\underline{LaTex Project}}
говорит, что никаких проблем при работе быть не может, аргументируя это следующим:\\

\hline
\begin{flushright}
Например, возьмём следующий документ:\\

\begin{flushleft}
\textit{
    \hspace{1cm} Cartesian closed categories and the price of eggs\\
    \hspace{1cm} Jane Doe\\
    \hspace{1cm} September 1994\\
    \\[0.5cm]\\
    \hspace{1cm} Hello world!}
\end{flushleft}

Для набора такого текста в большинстве систем вёрстки или текстовых процессорах автору нужно разметить текст,
например, выбрать шрифт Times Roman величиной 18 пунктов для заголовка, шрифт Times Italic 12 пунктов для имени и так далее.
Это приводит к двум результатам: авторы тратят время на оформление и появляется много плохо оформленных документов!\\
\\[0.5cm]\\
LaTeX основан на идее, что оформление документа лучше оставить дизайнерам, а авторы пусть сконцентрируются на написании текста.
Так что в LaTeX вы введёте такой документ следующим образом:\\

    \begin{verbatim}
        \documentclass{article}
        \title{Cartesian closed categories and the price of eggs}
        \author{Jane Doe}
        \date{September 1994}
        \begin{document}
        \maketitle
        Hello world!
        \end{document}
    \end{verbatim}

\end{flushright}
\hline

\begin{flushleft}
Но проблема в том, что большинство редакторов текста имеют свои собственные настройки по умолчанию, и для обычного пользователя они гораздо понятнее и проще.\\
То есть я это к чему - набрать просто\\
\hspace{1cm} \large Заголовок \normalsize \\
и нажать на пару кнопок оформления по завершении набора документа
проще, чем набрать:\\
\hspace{1cm} \large $\backslash$title\{Заголовок\} \normalsize \\
Как минимум поэтому LaTeX остается инструментом написания статей для ученых и их студентов, и не идет дальше.
\end{flushleft}

\end{document}