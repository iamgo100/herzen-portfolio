% Шаблон из ЛР2 (ЛР6)
\documentclass[a4paper,12pt]{article}
\usepackage[T2A]{fontenc}
\usepackage[utf8]{inputenc}
\usepackage[english,russian]{babel}
\usepackage{amsmath,amsfonts,amssymb,amsthm,mathtools}
\usepackage{wasysym}
\usepackage{hyperref}

\author{Елкина Галя, 2пг}
\title{Работа с текстом в \LaTeX{}}
\date{\today}

\begin{document}

\maketitle
\newpage

\section*{Задание 2}

\textit{Проработать материал прилагаемой публикации стр. 17 - 48.
Сформулируйте 10 вопросов к прочитанному тексту.
Создайте документ, содержащий нумерованный список вопросов к прочитанному тексту.
Представьте файл pdf и tex.}

\begin{enumerate}
    \item Как понять, где нужно использовать новый абзац, а где просто перевод строки?
    \item Для чего нужен запрет разрыва строк или страниц?
    \item Как напечатать кавычки-елочки?
    \item Для чего нужна команда $\backslash$ldots?
    \item Зачем нужны лигатуры?
    \item Чем отличается написание документа LaTeX на английском от других языков?
    \item Какие есть кодировки шрифтов в \LaTeX?
    \item Как поставить тире перед диалогом в кирилице?
    \item Как создать оглавление?
    \item Для чего нужна команда $\backslash$tabular?
\end{enumerate}

\end{document}